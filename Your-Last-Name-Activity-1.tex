% Options for packages loaded elsewhere
\PassOptionsToPackage{unicode}{hyperref}
\PassOptionsToPackage{hyphens}{url}
%
\documentclass[
]{article}
\usepackage{amsmath,amssymb}
\usepackage{iftex}
\ifPDFTeX
  \usepackage[T1]{fontenc}
  \usepackage[utf8]{inputenc}
  \usepackage{textcomp} % provide euro and other symbols
\else % if luatex or xetex
  \usepackage{unicode-math} % this also loads fontspec
  \defaultfontfeatures{Scale=MatchLowercase}
  \defaultfontfeatures[\rmfamily]{Ligatures=TeX,Scale=1}
\fi
\usepackage{lmodern}
\ifPDFTeX\else
  % xetex/luatex font selection
\fi
% Use upquote if available, for straight quotes in verbatim environments
\IfFileExists{upquote.sty}{\usepackage{upquote}}{}
\IfFileExists{microtype.sty}{% use microtype if available
  \usepackage[]{microtype}
  \UseMicrotypeSet[protrusion]{basicmath} % disable protrusion for tt fonts
}{}
\makeatletter
\@ifundefined{KOMAClassName}{% if non-KOMA class
  \IfFileExists{parskip.sty}{%
    \usepackage{parskip}
  }{% else
    \setlength{\parindent}{0pt}
    \setlength{\parskip}{6pt plus 2pt minus 1pt}}
}{% if KOMA class
  \KOMAoptions{parskip=half}}
\makeatother
\usepackage{xcolor}
\usepackage[margin=1in]{geometry}
\usepackage{color}
\usepackage{fancyvrb}
\newcommand{\VerbBar}{|}
\newcommand{\VERB}{\Verb[commandchars=\\\{\}]}
\DefineVerbatimEnvironment{Highlighting}{Verbatim}{commandchars=\\\{\}}
% Add ',fontsize=\small' for more characters per line
\usepackage{framed}
\definecolor{shadecolor}{RGB}{248,248,248}
\newenvironment{Shaded}{\begin{snugshade}}{\end{snugshade}}
\newcommand{\AlertTok}[1]{\textcolor[rgb]{0.94,0.16,0.16}{#1}}
\newcommand{\AnnotationTok}[1]{\textcolor[rgb]{0.56,0.35,0.01}{\textbf{\textit{#1}}}}
\newcommand{\AttributeTok}[1]{\textcolor[rgb]{0.13,0.29,0.53}{#1}}
\newcommand{\BaseNTok}[1]{\textcolor[rgb]{0.00,0.00,0.81}{#1}}
\newcommand{\BuiltInTok}[1]{#1}
\newcommand{\CharTok}[1]{\textcolor[rgb]{0.31,0.60,0.02}{#1}}
\newcommand{\CommentTok}[1]{\textcolor[rgb]{0.56,0.35,0.01}{\textit{#1}}}
\newcommand{\CommentVarTok}[1]{\textcolor[rgb]{0.56,0.35,0.01}{\textbf{\textit{#1}}}}
\newcommand{\ConstantTok}[1]{\textcolor[rgb]{0.56,0.35,0.01}{#1}}
\newcommand{\ControlFlowTok}[1]{\textcolor[rgb]{0.13,0.29,0.53}{\textbf{#1}}}
\newcommand{\DataTypeTok}[1]{\textcolor[rgb]{0.13,0.29,0.53}{#1}}
\newcommand{\DecValTok}[1]{\textcolor[rgb]{0.00,0.00,0.81}{#1}}
\newcommand{\DocumentationTok}[1]{\textcolor[rgb]{0.56,0.35,0.01}{\textbf{\textit{#1}}}}
\newcommand{\ErrorTok}[1]{\textcolor[rgb]{0.64,0.00,0.00}{\textbf{#1}}}
\newcommand{\ExtensionTok}[1]{#1}
\newcommand{\FloatTok}[1]{\textcolor[rgb]{0.00,0.00,0.81}{#1}}
\newcommand{\FunctionTok}[1]{\textcolor[rgb]{0.13,0.29,0.53}{\textbf{#1}}}
\newcommand{\ImportTok}[1]{#1}
\newcommand{\InformationTok}[1]{\textcolor[rgb]{0.56,0.35,0.01}{\textbf{\textit{#1}}}}
\newcommand{\KeywordTok}[1]{\textcolor[rgb]{0.13,0.29,0.53}{\textbf{#1}}}
\newcommand{\NormalTok}[1]{#1}
\newcommand{\OperatorTok}[1]{\textcolor[rgb]{0.81,0.36,0.00}{\textbf{#1}}}
\newcommand{\OtherTok}[1]{\textcolor[rgb]{0.56,0.35,0.01}{#1}}
\newcommand{\PreprocessorTok}[1]{\textcolor[rgb]{0.56,0.35,0.01}{\textit{#1}}}
\newcommand{\RegionMarkerTok}[1]{#1}
\newcommand{\SpecialCharTok}[1]{\textcolor[rgb]{0.81,0.36,0.00}{\textbf{#1}}}
\newcommand{\SpecialStringTok}[1]{\textcolor[rgb]{0.31,0.60,0.02}{#1}}
\newcommand{\StringTok}[1]{\textcolor[rgb]{0.31,0.60,0.02}{#1}}
\newcommand{\VariableTok}[1]{\textcolor[rgb]{0.00,0.00,0.00}{#1}}
\newcommand{\VerbatimStringTok}[1]{\textcolor[rgb]{0.31,0.60,0.02}{#1}}
\newcommand{\WarningTok}[1]{\textcolor[rgb]{0.56,0.35,0.01}{\textbf{\textit{#1}}}}
\usepackage{longtable,booktabs,array}
\usepackage{calc} % for calculating minipage widths
% Correct order of tables after \paragraph or \subparagraph
\usepackage{etoolbox}
\makeatletter
\patchcmd\longtable{\par}{\if@noskipsec\mbox{}\fi\par}{}{}
\makeatother
% Allow footnotes in longtable head/foot
\IfFileExists{footnotehyper.sty}{\usepackage{footnotehyper}}{\usepackage{footnote}}
\makesavenoteenv{longtable}
\usepackage{graphicx}
\makeatletter
\def\maxwidth{\ifdim\Gin@nat@width>\linewidth\linewidth\else\Gin@nat@width\fi}
\def\maxheight{\ifdim\Gin@nat@height>\textheight\textheight\else\Gin@nat@height\fi}
\makeatother
% Scale images if necessary, so that they will not overflow the page
% margins by default, and it is still possible to overwrite the defaults
% using explicit options in \includegraphics[width, height, ...]{}
\setkeys{Gin}{width=\maxwidth,height=\maxheight,keepaspectratio}
% Set default figure placement to htbp
\makeatletter
\def\fps@figure{htbp}
\makeatother
\setlength{\emergencystretch}{3em} % prevent overfull lines
\providecommand{\tightlist}{%
  \setlength{\itemsep}{0pt}\setlength{\parskip}{0pt}}
\setcounter{secnumdepth}{-\maxdimen} % remove section numbering
\ifLuaTeX
  \usepackage{selnolig}  % disable illegal ligatures
\fi
\usepackage{bookmark}
\IfFileExists{xurl.sty}{\usepackage{xurl}}{} % add URL line breaks if available
\urlstyle{same}
\hypersetup{
  pdftitle={Learning-R},
  pdfauthor={Jino Surendra},
  hidelinks,
  pdfcreator={LaTeX via pandoc}}

\title{Learning-R}
\author{Jino Surendra}
\date{2025-04-04}

\begin{document}
\maketitle

This is an activity created by Dr.~Antonio Paez, from his Session 2
Notes.

\section{My Goal to Become an Internal
Doctor}\label{my-goal-to-become-an-internal-doctor}

\emph{The following is an AI-created (ChatGPT by OpenAI) and
human-refined paragraph.}

I'm finishing my first year of university as a \textbf{Life Sciences
student}, and I've become really motivated to pursue a career in
\textbf{internal medicine}. This year, I've taken courses like:

\begin{itemize}
\tightlist
\item
  BIO 1A03
\item
  CHEM 1A03 and CHEM 1AA3
\item
  PHYSICS 1A03
\item
  MATH 1LS3
\end{itemize}

These courses helped me understand \emph{how the body works}, and it
made me realize how much I \emph{enjoy learning about health and
disease}. What I like most about internal medicine is how doctors work
with \emph{adult patients} to figure out what's going on when
something's wrong and help manage long-term conditions. I think it's
amazing how they \textbf{use both science and problem-solving} to help
people feel better and stay healthy. I also like that internal medicine
doctors get to build strong relationships with their patients over time.
I want to be someone who \textbf{listens, understands, and makes a real
difference in people's lives}. As I continue through university, I plan
to get involved in \emph{volunteering, research, and maybe even shadow
doctors} so I can learn more about the field and prepare for medical
school. Becoming a doctor is a \textbf{big goal}, but it's one that
feels right for me, and I'm excited to keep working toward it.

\section{Favorites}\label{favorites}

\subsection{Favorite Music}\label{favorite-music}

Below are a list of my \textbf{favorite songs} at the moment:

\begin{enumerate}
\def\labelenumi{\arabic{enumi}.}
\tightlist
\item
  Joro - \emph{Wizkid}
\item
  Don't Go Breaking My Heart - \emph{Elton John, Kiki Dee}
\item
  GOOD CREDIT - \emph{Playboi Carti feat. Kendrick Lamar}
\item
  Magic Don Juan (Princess Diana) - \emph{Metro Boomin, Future}
\item
  Fight For Your Right - \emph{Beastie Boys}
\end{enumerate}

As you can see, I listen to pretty much any music that has been
released.

\subsection{Favorite Equation}\label{favorite-equation}

My favorite equation, I would say in my life, is the plain and simple
Einstein's mass-energy equivalence formula: \(E = mc^2\).

When I was young, I always looked at this equation as a super complex
equation that only the best of scientists knew how to use. But, when I
got to Grade 11, I learned how simple it is, how easy it is to apply and
use when needed.

\subsection{Favorite Artists}\label{favorite-artists}

Below is a table of my favorite artists along with one of their most
notable achievements:

\begin{longtable}[]{@{}
  >{\raggedright\arraybackslash}p{(\columnwidth - 2\tabcolsep) * \real{0.5000}}
  >{\raggedright\arraybackslash}p{(\columnwidth - 2\tabcolsep) * \real{0.5000}}@{}}
\toprule\noalign{}
\begin{minipage}[b]{\linewidth}\raggedright
Artist Name
\end{minipage} & \begin{minipage}[b]{\linewidth}\raggedright
Achievement
\end{minipage} \\
\midrule\noalign{}
\endhead
\bottomrule\noalign{}
\endlastfoot
Metro Boomin & Best Rap Song (\emph{2024}) \\
21 Savage & Album of The Year (\emph{2024}) \\
Post Malone & Best Country Album (\emph{2025}) \\
Young Thug & Best Rap Performance (\emph{2023}) \\
WizKid & Songwriter for First Song to Reach 1,000,000,000 Streams on
Spotify \\
\end{longtable}

\section{A Chunk of Code}\label{a-chunk-of-code}

\begin{Shaded}
\begin{Highlighting}[]
\CommentTok{\# Below is a chunk of code that plots out a graph of y = 2\^{}x.}

\CommentTok{\# Define x values (5 points).}
\CommentTok{\# 0 is counted as the first value in R.}
\NormalTok{x }\OtherTok{\textless{}{-}} \FunctionTok{c}\NormalTok{(}\DecValTok{0}\NormalTok{, }\DecValTok{1}\NormalTok{, }\DecValTok{2}\NormalTok{, }\DecValTok{3}\NormalTok{, }\DecValTok{4}\NormalTok{)}

\CommentTok{\# Caclulate y = 2\^{}x.}
\NormalTok{y }\OtherTok{\textless{}{-}} \DecValTok{2}\SpecialCharTok{\^{}}\NormalTok{x}

\CommentTok{\# Plot the points}
\CommentTok{\# Type "b" refers to the type of graph, this connects the points and shows dots.}
\CommentTok{\# Color can also be chosen, as purple is my favorite color, I have chosen it.}
\CommentTok{\# Expression refers to the title which currently shows the equation of the graph.}
\FunctionTok{plot}\NormalTok{(x, y, }\AttributeTok{type =} \StringTok{"b"}\NormalTok{, }\AttributeTok{col =} \StringTok{"purple"}\NormalTok{, }\AttributeTok{pch =} \DecValTok{20}\NormalTok{,}
     \AttributeTok{main =} \FunctionTok{expression}\NormalTok{(y }\SpecialCharTok{==} \DecValTok{2}\SpecialCharTok{\^{}}\NormalTok{x),}
     \AttributeTok{xlab =} \StringTok{"x"}\NormalTok{, }\AttributeTok{ylab =} \StringTok{"y"}\NormalTok{)}

\CommentTok{\# Adds grid lines to the graph for clarity}
\FunctionTok{grid}\NormalTok{()}
\end{Highlighting}
\end{Shaded}

\includegraphics{Your-Last-Name-Activity-1_files/figure-latex/unnamed-chunk-1-1.pdf}

\end{document}
